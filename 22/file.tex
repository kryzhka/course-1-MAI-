\documentclass[a5paper,14pt]{book}
\usepackage[OT1]{fontenc}
\usepackage[utf8]{inputenc}
\usepackage[english, russian]{babel}


\usepackage[left=1.5cm,right=1.5cm,top=2cm,bottom=0.5cm,bindingoffset=0cm]{geometry}
\usepackage{setspace}
\linespread{0.6}
\let\emph\textit
\usepackage[symbol*]{footmisc}
\usepackage{amsmath, amssymb}
\usepackage{wasysym}
\usepackage{cases}
\begin{document}
\markboth{\small{\textsc{ уравнения математической физики \hspace{1cm} [гл. XVIII}}}
{\small{\textsc{\$7] \hspace{1cm} распространение тепла в неограниченном стержне}}}

\setcounter{page}{382}

Это преобразование интегралл сделано путем подстановок
$$
 a\lambda\sqrt{t}=z, \qquad \frac{a-x}{a\sqrt{t}}=\beta \eqno{(14)}
$$
Обозначим
$$
	K(\beta)=\int_0^{+\infty} e^{-z^2} \cos{\beta z}dz. \eqno{(15)}
$$
Дифференцируя \footnote{)Возможность дифференцирования легко обосновываетя.}), получаем
$$
	K'(\beta) = - \int_0^{+\infty}e^{-z^2} z \sin{\beta z}dz.
$$
Интегрируя по частям, найдем
$$
	K'(\beta) = \frac{1}{2} [e^{-z^2} \sin{\beta z}]_0^{+\infty} +\infty 0 - \frac{\beta}{2} \int_0^{+\infty} e^{-z^2} \cos{\beta z}dz
$$
или
$$
	K'(\beta) =Ce^{-\frac{\beta^2}{4}}. \eqno{(16)}
$$
Определим постоянную C. Из (15) сдедует
$$
	K(0)=\int_0^{+\infty}e^{-z^2}dz=\frac{\sqrt{\pi}}{2}
$$
(см. $\mathsection$ 5 гл. XIV). Следовательно, в равенстве (16) должно быть
$$
	C=\frac{\sqrt{\pi}}{2}.
$$
Итак
$$
	K(\beta)=\frac{\sqrt{\pi}}{2} e^{-\frac{-\beta^2}{4}}. \eqno{(17)}
$$
Значение (17) интегралла (15) подставляем в (13):
$$
	\int_0^{+\infty}e^{-a^2\lambda^2t}\cos{\lambda}(a-x)d\lambda=\frac{1}{a\sqrt{t}}\frac{sqrt{\pi}}{2}e^{-\frac{\beta^2}{4}}.
$$
Подставляя вместо $\beta$ его выражение (14), окончательно получаем

\newpage
значение интегралла (13):
$$
	\int_0^{+\infty}e^{-a^2\lambda^2t}cos{\lambda}(a-x)d\lambda=\frac{1}{2a}\sqrt{\frac{\pi}{t}e}^{-\frac{(a-x)^2}{4a^2t}}.\eqno{(18)}
$$
Подставив это выражение интеграла в решение (12), окончательно получим:
$$
	u(x,t)=\frac{1}{2a\sqrt{\pi t}}\int_{-\infty}^{+\infty}\varphi(\alpha)e^{-\frac{(a-x)^2}{4a^2t}}d\alpha. \eqno {(19)}
$$

Эта формула, называемая \textit{интегралом Пауссона}, представляет собой решение поставленной задачи о распространении тепла в неограниченном стержне.

\textsc{Замечание.} Можно доказать, что функция u(x,t), определенная интегралом (19), является решением уравнения (1) и удовлетворяет условию (2), если функция $\varphi(x)$ ограничена на бесконечном интервале $(-\infty,+\infty)$.

Установим физический смысл формулы(19). Рассмотрим функцию
\begin{equation*}
\text{$\varphi^*(x)=$}
\begin{cases}
	0 &\text{при$ -\infty < x < x_0 $}\\
	\text{$\varphi(x)$} &\text{при $ x_0 \leq x \leq x_0 +\Delta x$}\\
	0 &\text{при $x_0+\Delta x < x < +\infty .$}
\end{cases}
\end{equation*}
Тогда функция
$$
	u^*(x,t)=\frac{1}{2a\sqrt{\pi t}} \int_{-\infty}^{+\infty}\varphi(\alpha)e^{-\frac{(a-x)^2}{4a^2t}}d\alpha \eqno {(21)}
$$
Есть решение уравнения (1), принимающее при t=0 значение $\varphi *(x)$. Принимая во внимание (20), можем написать
$$
	u^*(x,t)=\frac{1}{2a\sqrt{\pi t}} \int_{x_0}^{x_0+\Delta x}\varphi(\alpha)e^{-\frac{(a-x)^2}{4a^2t}}d\alpha.
$$
Применив теорему о среднем к последующему интегралу, получим
$$
u^*(x,t)=\frac{\varphi(\xi)\Delta x}{2a\sqrt{\pi t}}e^{-\frac{(\xi-x)^2}{4a^2t}}, x_0 < \xi < x_0 + \Delta x. \eqno{(22)}
$$
Формула (22) дает значение температуры в точке стержня в любой момент времени, если при t=0 всюду в стержне температура $u^*=0$, кроме отрезка $[x_0,x_0+\Delta x]$, где она равна $\varphi(x)$. Сумма температур вида (22) и дает решение (19). Заметим, что если $\rho$- линейная плотность стержня, c - теплоемкость материала, то количество тепла в элементе $[x_0, x_0+\Delta x]$ при t=0 будет
$$
\Delta Q \approx  \varphi(\xi)\Delta x \rho c. \eqno{(23)}
$$
\end{document}